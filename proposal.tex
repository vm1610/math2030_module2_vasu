\documentclass[12pt]{article}

%% Packages
\usepackage{amsmath, amssymb, graphicx, hyperref}
\usepackage[a4paper, margin=1in]{geometry}

\hypersetup{
    colorlinks=true,
    urlcolor=blue,
    linkcolor=black
}

%% Title Page
\title{
    \includegraphics[width=0.25\textwidth]{mun_logo.pdf}\\[1em]
    \textbf{Project Proposal: Learning Chaos – Visualizing Sensitivity in the Logistic Map}\\
    \vspace{0.5em}
    \large MATH 2030 
}
\author{
    \textbf{Vasu Manocha} \\
    \vspace{0.3em}
    Instructor: Dr. Alexander Bihlo
}

\date{October 18, 2025}

\begin{document}
\maketitle
\thispagestyle{empty}
\clearpage

\setcounter{page}{1}


\section{Topic Description}
The goal of this project is to explore how complex, chaotic behaviour can emerge from an extremely simple mathematical rule. Specifically, the project will focus on the \textbf{logistic map}, defined by
\[
x_{n+1} = r\,x_n(1 - x_n),
\]
where \(0 < x_n < 1\) represents a normalized population and \(r\) is a control parameter. Despite the simplicity of this iterative equation, it exhibits rich dynamics—period-doubling, bifurcations, and eventually fully chaotic motion—as \(r\) increases.

This phenomenon illustrates one of the central ideas of nonlinear dynamics: that deterministic systems can produce apparently random outcomes. The project aims to visualize these transitions, quantify the onset of chaos, and provide mathematical as well as computational insight into why small parameter or initial-condition changes lead to large, unpredictable effects.

This investigation connects fundamental mathematics (fixed points, stability, iteration) with computational experimentation and data visualization in Python. The resulting report will combine theoretical background with visual evidence to communicate how chaos can arise in even the simplest nonlinear systems.


\section{Research Questions}
\begin{itemize}
    \item How does the logistic map transition from stable to chaotic behaviour as the control parameter \(r\) increases?
    \item Can we estimate the \textbf{Feigenbaum constant} numerically by identifying successive period-doubling bifurcations?
    \item How can we demonstrate and quantify the \textbf{sensitive dependence on initial conditions} that defines chaos?
\end{itemize}

\section{Methodology}

\begin{itemize}
     

    \item Implement the logistic map in Python and simulate iterations for a range of \(r\) values (\(2.5 \leq r \leq 4.0\)).
    \item Generate and plot the \textbf{bifurcation diagram} to visualize the transition from periodic to chaotic behaviour.
    \item Identify approximate bifurcation points and compute successive ratios to estimate the Feigenbaum constant.
    \item Run simulations with nearly identical initial conditions (e.g., \(x_0 = 0.5000\) and \(x_0 = 0.5001\)) to illustrate divergence.
\end{itemize}
\section{Version Control and Tools}
% (WORK IN PROGRESS)

\section{References}
% ( WORK IN PROGRESS)

\end{document}
